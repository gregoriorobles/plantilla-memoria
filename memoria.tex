%%%%%%%%%%%%%%%%%%%%%%%%%%%%%%%%%%%%%%%%%%%%%%%%%%%%%%%%%%%%%%%%%%%%%%%%%%%%%%%%
%% Plantilla de memoria en LaTeX para la EIF - Universidad Rey Juan Carlos
%%
%% Por Gregorio Robles <grex arroba gsyc.urjc.es>
%%     Grupo de Sistemas y Comunicaciones
%%     Escuela de Ingeniería de Fuenlabrada
%%     Universidad Rey Juan Carlos
%% (muchas ideas tomadas de Internet, colegas del GSyC, antiguos alumnos...
%%  etc. Muchas gracias a todos)
%%
%% La última versión de esta plantilla está siempre disponible en:
%%     https://github.com/gregoriorobles/plantilla-memoria
%%
%% Para obtener PDF, ejecuta en la shell:
%%   make
%% (las imágenes deben ir en PNG o JPG)

%%%%%%%%%%%%%%%%%%%%%%%%%%%%%%%%%%%%%%%%%%%%%%%%%%%%%%%%%%%%%%%%%%%%%%%%%%%%%%%%

\documentclass[a4paper, 12pt]{book}
%\usepackage[T1]{fontenc}

\usepackage[a4paper, left=2.5cm, right=2.5cm, top=3cm, bottom=3cm]{geometry}
\usepackage{times}
\usepackage[utf8]{inputenc}
\usepackage[spanish]{babel} % Comenta esta línea si tu memoria es en inglés
\usepackage{url}
%\usepackage[dvipdfm]{graphicx}
\usepackage{graphicx}
\usepackage{float}  %% H para posicionar figuras
\usepackage[nottoc, notlot, notlof, notindex]{tocbibind} %% Opciones de índice
\usepackage{latexsym}  %% Logo LaTeX

\title{Memoria del Proyecto}
\author{Nombre del autor}

\renewcommand{\baselinestretch}{1.5}  %% Interlineado

\begin{document}

\renewcommand{\refname}{Bibliografía}  %% Renombrando
\renewcommand{\appendixname}{Apéndice}


%%%%%%%%%%%%%%%%%%%%%%%%%%%%%%%%%%%%%%%%%%%%%%%%%%%%%%%%%%%%%%%%%%%%%%%%%%%%%%%%
% PORTADA

\begin{titlepage}
\begin{center}
\includegraphics[scale=0.6]{img/URJ_logo_Color_POS.png}

\vspace{1.75cm}

\LARGE
ESCUELA DE INGENIERÍA DE FUENLABRADA
\vspace{1cm}

\LARGE
GRADO EN INGENIERÍA DE TECNOLOGÍAS DE TELECOMUNICACIÓN

\vspace{1cm}
\LARGE
\textbf{TRABAJO FIN DE GRADO}

\vspace{2cm}

\Large
EVOLUCIÓN DE LA PARTICIPACIÓN VOLUNTARIA DE PROYECTOS DE SOFTWARE LIBRE:
EVIDENCIAS DE DEBIAN

\vspace{2cm}

\large
Autor : Pablo Cabeza Portalo \\
Tutor : Dr. Gregorio Robles\\

\vspace{1cm}

\large
Curso académico 2023/2024

\end{center}
\end{titlepage}

\newpage
\mbox{}
\thispagestyle{empty} % para que no se numere esta pagina



%%%%%%%%%%%%%%%%%%%%%%%%%%%%%%%%%%%%%%%%%%%%%%%%%%%%%%%%%%%%%%%%%%%%%%%%%%%%%%%%
%%%% Para firmar
\clearpage
\pagenumbering{gobble}
\chapter*{}

\vspace{-4cm}
\begin{center}
\LARGE
\textbf{Trabajo Fin de Grado}

\vspace{1cm}
\large
Evolución de la Participación Voluntaria en Proyectos de Software Libre:
Evidencia de Debian

\vspace{1cm}
\large
\textbf{Autor :} Pablo Cabeza Portalo \\
\textbf{Tutor :} Dr. Gregorio Robles

\end{center}

\vspace{1cm}
La defensa del presente Proyecto Fin de Carrera se realizó el día \qquad$\;\,$ de \qquad\qquad\qquad\qquad \newline de 2024, siendo calificada por el siguiente tribunal:


\vspace{0.5cm}
\textbf{Presidente:}

\vspace{1.2cm}
\textbf{Secretario:}

\vspace{1.2cm}
\textbf{Vocal:}


\vspace{1.2cm}
y habiendo obtenido la siguiente calificación:

\vspace{1cm}
\textbf{Calificación:}


\vspace{1cm}
\begin{flushright}
Fuenlabrada, a \qquad$\;\,$ de \qquad\qquad\qquad\qquad de 2024
\end{flushright}

%%%%%%%%%%%%%%%%%%%%%%%%%%%%%%%%%%%%%%%%%%%%%%%%%%%%%%%%%%%%%%%%%%%%%%%%%%%%%%%%
%%%% Dedicatoria

\chapter*{}
\pagenumbering{Roman} % para comenzar la numeracion de paginas en numeros romanos
\begin{flushright}
\textit{Dedicado a \\
mi familia, mis amigos y a mi pareja.}
\end{flushright}

%%%%%%%%%%%%%%%%%%%%%%%%%%%%%%%%%%%%%%%%%%%%%%%%%%%%%%%%%%%%%%%%%%%%%%%%%%%%%%%%
%%%% Agradecimientos

\chapter*{Agradecimientos}
%\addcontentsline{toc}{chapter}{Agradecimientos} % si queremos que aparezca en el índice
\markboth{AGRADECIMIENTOS}{AGRADECIMIENTOS} % encabezado 

Este es el fin de una de las etapas más importantes de mi vida y por ello quiero agradecer a varias personas que me han acompañado en este proceso.
Primero quiero agradecer a mi madre Antonia todo el apoyo que me ha dado en cada una de mis decisiones, tanto personales como académicas. Eres quien me impulsa a lograr mis sueños.
Segundo a mi padre Manuel que siempre ha sido mi ejemplo a seguir. Me ha aportado valores fundamentales como el trabajo y el esfuerzo que han sido imprescindibles para abordar esta carrera.
Gracias a los dos por darme la vida.
También agradecer a mi tutor, Gregorio, por darme este proyecto, tener paciencia  y ayudarme en todo cuanto pudo para desarrollarlo.
Por último, quiero agradecer a esa persona tan especial que conocí en esta universidad y que me dio la confianza y la fuerza necesaria para lograr mis metas. 
Paula, gracias por ser mi compañera de viaje y por aparecer en el momento que mas lo necesitaba. Hemos afrontado este desafío juntos y sin ti no hubiera sido igual.


%%%%%%%%%%%%%%%%%%%%%%%%%%%%%%%%%%%%%%%%%%%%%%%%%%%%%%%%%%%%%%%%%%%%%%%%%%%%%%%%
%%%% Resumen

\chapter*{Resumen}
%\addcontentsline{toc}{chapter}{Resumen} % si queremos que aparezca en el índice
\markboth{RESUMEN}{RESUMEN} % encabezado

Aquí viene un resumen del proyecto.
Ha de constar de tres o cuatro párrafos, donde se presente de manera clara y concisa de qué va el proyecto. 
Han de quedar respondidas las siguientes preguntas:

\begin{itemize}
  \item ¿De qué va este proyecto? ¿Cuál es su objetivo principal?
  \item ¿Cómo se ha realizado? ¿Qué tecnologías están involucradas?
  \item ¿En qué contexto se ha realizado el proyecto? ¿Es un proyecto dentro de un marco general?
\end{itemize}

Lo mejor es escribir el resumen al final.

%%%%%%%%%%%%%%%%%%%%%%%%%%%%%%%%%%%%%%%%%%%%%%%%%%%%%%%%%%%%%%%%%%%%%%%%%%%%%%%%
%%%% Resumen en inglés

\chapter*{Summary}
%\addcontentsline{toc}{chapter}{Summary} % si queremos que aparezca en el índice
\markboth{SUMMARY}{SUMMARY} % encabezado

Here comes a translation of the ``Resumen'' into English. 
Please, double check it for correct grammar and spelling.
As it is the translation of the ``Resumen'', which is supposed to be written at the end, this as well should be filled out just before submitting.


%%%%%%%%%%%%%%%%%%%%%%%%%%%%%%%%%%%%%%%%%%%%%%%%%%%%%%%%%%%%%%%%%%%%%%%%%%%%%%%%
%%%%%%%%%%%%%%%%%%%%%%%%%%%%%%%%%%%%%%%%%%%%%%%%%%%%%%%%%%%%%%%%%%%%%%%%%%%%%%%%
% ÍNDICES %
%%%%%%%%%%%%%%%%%%%%%%%%%%%%%%%%%%%%%%%%%%%%%%%%%%%%%%%%%%%%%%%%%%%%%%%%%%%%%%%%

% Las buenas noticias es que los índices se generan automáticamente.
% Lo único que tienes que hacer es elegir cuáles quieren que se generen,
% y comentar/descomentar esa instrucción de LaTeX.

%%%% Índice de contenidos
\tableofcontents 
%%%% Índice de figuras
\cleardoublepage
%\addcontentsline{toc}{chapter}{Lista de figuras} % para que aparezca en el indice de contenidos
\listoffigures % indice de figuras
%%%% Índice de tablas
%\cleardoublepage
%\addcontentsline{toc}{chapter}{Lista de tablas} % para que aparezca en el indice de contenidos
%\listoftables % indice de tablas


%%%%%%%%%%%%%%%%%%%%%%%%%%%%%%%%%%%%%%%%%%%%%%%%%%%%%%%%%%%%%%%%%%%%%%%%%%%%%%%%
%%%%%%%%%%%%%%%%%%%%%%%%%%%%%%%%%%%%%%%%%%%%%%%%%%%%%%%%%%%%%%%%%%%%%%%%%%%%%%%%
% INTRODUCCIÓN %
%%%%%%%%%%%%%%%%%%%%%%%%%%%%%%%%%%%%%%%%%%%%%%%%%%%%%%%%%%%%%%%%%%%%%%%%%%%%%%%%

\cleardoublepage
\chapter{Introducción}
\label{sec:intro} % etiqueta para poder referenciar luego en el texto con ~\ref{sec:intro}
\pagenumbering{arabic} % para empezar la numeración de página con números

Vivimos en un mundo totalmente digitalizado en el que la presencia de ordenadores está a la orden del día. Con esto no nos referimos únicamente a ordenadores de escritorio. A diario interactuamos con una amplia gama de dispositivos compuestos por ordenadores camuflados. Desde electrodomésticos inteligentes pasando por coches modernos hasta en tarjetas de crédito. Estos tienen incorporados ordenadores pequeños pero potentes que realizan una serie de tareas para el beneficio y la mejora de la vida humana.

En la mayoría de estos ordenadores de uso cotidiano, como portátiles o móviles, se aloja un Sistema Operativo. Un Sistema Operativo es el “intermediario” entre el usuario y el hardware del ordenador a partir de software. Gestiona los recursos del hardware proporcionando una interfaz al usuario. De esta forma pueden interactuar con dicho ordenador. 
Algunos de los sistemas operativos más usados son: iOS, Android, macOS, Microsoft Windows o Linux. 

Linux es un sistema operativo de código abierto el cual es gratuito para cualquier usuario que quiera adoptarlo en su computadora. Este consta de muchas distribuciones. Las distribuciones son versiones del Sistema Operativo de Linux desarrolladas por diferentes individuos, equipos o empresas para mejorar la experiencia de los usuarios.

Una de estas distribuciones es “Debian” y es sobre la que tratará este estudio.
Debian es un Sistema Operativo que trabaja con el Kernel (núcleo) de Linux y ha ido aportando distintas versiones desde 1993. 
Treinta años después nos preguntamos ciertas cosas como, ¿Los individuos que trabajaban en las primeras versiones siguen actualizando Debian? ¿Se trabaja individualmente o por equipos? ¿Cuántos paquetes sacan en cada versión? ¿Qué ocurre con ellos? Todo esto lo veremos a continuación.



\section{Contexto}
\label{sec:contexto}



\subsection{Proyecto Debian}
\label{subsec:proyecto debian}

El Proyecto Debian está formado por un grupo de voluntarios a nivel mundial que trabajan para producir una distribución del Sistema Operativo Linux basada en "software libre".

Con el término "software libre" no nos referimos a su coste. Este va enfocado a la "libertad real" dentro del software, es decir, "software de código abierto".
Esto significa que cualquier usuario puede acceder al código fuente para estudiarlo, revisarlo, modificarlo o distribuirlo sin restricción alguna.

Debian es la distribución de Linux más relevante sin fines comerciales.
En su comienzo, fue la única abierta a la participación de diferentes usuarios que quisieran aportar al proyecto con su trabajo.

Con el tiempo fue asentando un gran conjunto de directrices y procedimientos para el empaquetamiento y distribución de software. Esto les sirvió para poder alcanzar los estándares de calidad requeridos y con ello asegurar su buen funcionamiento.

\subsection{Versiones Debian}
\label{subsec:versiones debian}

Debian está conformado por varias versiones desde 1993 las cuales explicaremos a continuación:

\begin{itemize}

	\item \textbf {Versiones 0.x:} estas versiones fueron las primeras y mas rudimentarias pero dieron lugar a la creación de Debian.
	\begin{itemize}
		\item \textbf {Debian 0.01 hasta 0.90.}
		\item \textbf {Debian 0.91:} disponía de un sencillo sistema de empaquetamiento que permitía instalar y desinstalar paquetes.
		\item \textbf {Debian 0.93R5:}  se asignaron responsabilidades de cada paquete a cada uno de los desarrolladores. Se comenzo a usar el administrador de paquetes \textbf {dpkg} para la instalación de paquetes después de la instalación del sistema.
		base.
		\item \textbf {Debian 1.0:} esta versión nunca fue publicada debido a una confusión al distribuir una versión en desarrollo con el nombre equivocado de Debian 1.0 que daría problemas en ejecución.
		\item \textbf {Debian 1.1 Buzz: } es la primera versión de Debian con un nombre en clave sacado de las peliculas de \textbf {"Toy Story"}.
		\item \textbf {Debian 1.2 Rex: } esta versión estaba completamente en formato \textbf{ELF} y usaba el núcleo (kernel) Linux 2.0.
		
		El formato\textbf{ELF} (Executable and Linkable Format) es un estándar. Se usa en sistemas operativos tipo \textbf{UNIX} (como Linux). Sirve para organizar y manejar archivos ejecutables, bibliotecas compartidas y otros objetos binarios.
		\item \textbf {Debian 1.3 Bo}.
		\item \textbf {Debian 2.0 Hamm: }fue la primera versión multiplataforma de Debian. Agregó soporte para arquitecturas de la serie \textbf{Motorola 68000}.
		\item \textbf {Debian 2.2 Potato: }agregó soporte para las arquitecturas  PowerPC y ARM (CPU's de arquitectura RISC creadas por diferentes empresas).
		\item \textbf {Debian 3.0 Woody: }se agregaron más arquitecturas a esta versión y fue la primera en usar \textbf {software criptográfico}. Este se usa para codificar información y mantener la transferencia segura de datos.
		\item \textbf {Debian 3.1 Sarge: }incluye un nuevo instalador llamado \textbf {debian-installer}. Contiene detección automática de hardware, instalación sin supervisión y está traducido a más de treinta idiomas.
		\item \textbf {Debian 4.0 Etch: }se añadieron mejoras como un instalador gráfico o la verificación criptográfica de los paquetes descargados entre otras.
		\item \textbf {Debian 5.0 Lenny: }añadió la arquitectura \textbf {ARM EABI} para dar soporte a los nuevos procesadores \textbf {ARM}.
		\item \textbf {Debian 6.0 Squeeze: }con esta versión fue la primera vez que una distribución de Linux se extendía para permitir también el uso de un núcleo no Linux.
		\item \textbf {Debian 7.0 Wheezy: }se introdujo el soporte de \textbf {multiarquitectura}. Esto permitía que los usuarios instalaran en una misma máquina paquetes de múltiples arquitecturas.
		\item \textbf {Debian 8 Jessie: }trajo importantes mejoras de seguridad, como un nuevo kernel que solucionaba varias vulnerabilidades (como \textbf {ataques de enlace simbólico}).
		\item \textbf {Debian 9 Stretch: }se introdujeron paquetes para la depuración a través de un repositorio nuevo en el archivo. Facilitaría el proceso de depuración y solución de problemas relacionados con esos paquetes.
		\item \textbf {Debian 10 Buster: } incluyó por primera vez un marco de control de acceso obligatorio. Restringe las acciones que pueden realizar los programas, limitando su acceso a ciertos recursos del sistema, como archivos, directorios, redes, etc.
		\item \textbf {Debian 11 Bullseye} introduce un nuevo paquete, ipp-usb, que utiliza el protocolo IPP-over-USB, independiente del fabricante y soportado por muchas impresoras actuales. Esto permite que un dispositivo USB sea tratado como un dispositivo de red.
		
		
	\end{itemize}
	
\end{itemize}

\section{Estructura de la memoria}
\label{sec:estructura}







En esta sección se debería introducir la estructura de la memoria. 

Así:


\begin{itemize}
  \item En el primer capítulo se hace una intro al proyecto.
  
  \item En el capítulo~\ref{chap:objetivos} (ojo, otra referencia automática) se muestran los objetivos del proyecto.
  
  \item A continuación se presenta el estado del arte en el capítulo~\ref{chap:estado}.
  
  \item \ldots
\end{itemize}





%%%%%%%%%%%%%%%%%%%%%%%%%%%%%%%%%%%%%%%%%%%%%%%%%%%%%%%%%%%%%%%%%%%%%%%%%%%%%%%%
%%%%%%%%%%%%%%%%%%%%%%%%%%%%%%%%%%%%%%%%%%%%%%%%%%%%%%%%%%%%%%%%%%%%%%%%%%%%%%%%
% OBJETIVOS %
%%%%%%%%%%%%%%%%%%%%%%%%%%%%%%%%%%%%%%%%%%%%%%%%%%%%%%%%%%%%%%%%%%%%%%%%%%%%%%%%

\cleardoublepage % empezamos en página impar
\chapter{Objetivos} % título del capítulo (se muestra)
\label{chap:objetivos} % identificador del capítulo (no se muestra, es para poder referenciarlo)

\section{Objetivo general} % título de sección (se muestra)
\label{sec:objetivo-general} % identificador de sección (no se muestra, es para poder referenciarla)

Aquí vendría el objetivo general en una frase:
Mi trabajo fin de grado consiste en crear de una herramienta de análisis de los comentarios jocosos en repositorios de software libre alojados en la plataforma GitHub.

Recuerda que los objetivos siempre vienen en infinitivo.


\section{Objetivos específicos}
\label{sec:objetivos-especificos}

Los objetivos específicos se pueden entender como las tareas en las que se ha desglosado el objetivo general.
Y, sí, también vienen en infinitivo.


\section{Planificación temporal}
\label{sec:planificacion-temporal}

A mí me gusta que aquí pongáis una descripción de lo que os ha llevado realizar el trabajo.
Hay gente que añade un diagrama de GANTT.
Lo importante es que quede claro cuánto tiempo llevas (tiempo natural, p.ej., 6 meses) y a qué nivel de esfuerzo (p.ej., principalmente los fines de semana).


%%%%%%%%%%%%%%%%%%%%%%%%%%%%%%%%%%%%%%%%%%%%%%%%%%%%%%%%%%%%%%%%%%%%%%%%%%%%%%%%
%%%%%%%%%%%%%%%%%%%%%%%%%%%%%%%%%%%%%%%%%%%%%%%%%%%%%%%%%%%%%%%%%%%%%%%%%%%%%%%%
% ESTADO DEL ARTE %
%%%%%%%%%%%%%%%%%%%%%%%%%%%%%%%%%%%%%%%%%%%%%%%%%%%%%%%%%%%%%%%%%%%%%%%%%%%%%%%%

\cleardoublepage
\chapter{Estado del arte}
\label{chap:estado}

Descripción de las tecnologías que utilizas en tu trabajo. 
Con dos o tres párrafos por cada tecnología, vale. 
Se supone que aquí viene todo lo que no has hecho tú.

Puedes citar libros, como el de Bonabeau et al., sobre procesos estigmérgicos~\cite{bonabeau:_swarm}. 
Me encantan los procesos estigmérgicos.
Deberías leer más sobre ellos.
Pero quizás no ahora, que tenemos que terminar la memoria para sacarnos por fin el título.
Nota que el \~ \ añade un espacio en blanco, pero no deja que exista un salto de línea. 
Imprescindible ponerlo para las citas.

Citar es importantísimo en textos científico-técnicos. 
Porque no partimos de cero.
Es más, partir de cero es de tontos; lo suyo es aprovecharse de lo ya existente para construir encima y hacer cosas más sofisticadas.
¿Dónde puedo encontrar textos científicos que referenciar?
Un buen sitio es Google Scholar\footnote{\url{http://scholar.google.com}}.
Por ejemplo, si buscas por ``stigmergy libre software'' para encontrar trabajo sobre software libre y el concepto de \emph{estigmergia} (¿te he comentado que me gusta el concepto de estigmergia ya?), encontrarás un artículo que escribí hace tiempo cuyo título es ``Self-organized development in libre software: a model based on the stigmergy concept''.
Si pulsas sobre las comillas dobles (entre la estrella y el ``citado por ...'', justo debajo del extracto del resumen del artículo, te saldrá una ventana emergente con cómo citar.
Abajo a la derecha, aparece un enlace BibTeX.
Púlsalo y encontrarás la referencia en formato BibTeX, tal que así:

{\footnotesize
\begin{verbatim}
@inproceedings{robles2005self,
  title={Self-organized development in libre software:
         a model based on the stigmergy concept},
  author={Robles, Gregorio and Merelo, Juan Juli\'an 
          and Gonz\'alez-Barahona, Jes\'us M.},
  booktitle={ProSim'05},
  year={2005}
}
\end{verbatim}
}

Copia el texto en BibTeX y pégalo en el fichero \texttt{memoria.bib}, que es donde están las referencias bibliográficas.
Para incluir la referencia en el texto de la memoria, deberás citarlo, como hemos hecho antes con~\cite{bonabeau:_swarm}, lo que pasa es que en vez de el identificador de la cita anterior (bonabeau:\_swarm), tendrás que poner el nuevo (robles2005self).
Compila el fichero \texttt{memoria.tex} (\texttt{pdflatex memoria.tex}), añade la bibliografía (\texttt{bibtex memoria.aux}) y vuelve a compilar \texttt{memoria.tex} (\texttt{pdflatex memoria.tex})\ldots y \emph{voilà} ¡tenemos una nueva cita~\cite{robles2005self}!

También existe la posibilidad de poner notas al pie de página, por ejemplo, una para indicarte que visite la página del GSyC\footnote{\url{http://gsyc.es}}.



\section{Sección 1} 
\label{sec:seccion1}

Hemos hablado de cómo incluir figuras.
Pero no hemos dicho nada de tablas.
A mí me gustan las tablas.
Mucho.
Aquí un ejemplo de tabla, la Tabla~\ref{tab:ejemplo} (siento ser pesado, pero nota cómo he puesto la referencia).

\begin{table}
 \begin{center}
  \begin{tabular}{ | l | c | r |} % tenemos tres colummnas, la primera alineada a la izquierda (l), la segunda al centro (c) y la tercera a la derecha (r). El | indica que entre las columnas habrá una línea separadora.
    \hline
    Uno & 2 & 3 \\ \hline % el hline nos da una línea vertical
    Cuatro & 5 & 6 \\ \hline
    Siete & 8 & 9 \\
    \hline
  \end{tabular}
  \caption{Ejemplo de tabla. Aquí viene una pequeña descripción (el \emph{caption}, el pie de tabla/figura) del contenido de la tabla. Si la tabla no es autoexplicativa, siempre viene bien aclararla aquí.}
  \label{tab:ejemplo}
 \end{center}
\end{table}

Hay un sitio en Internet donde puedes diseñar las tablas fácilmente y luego hacer un corta y pega del resultado en tu editor.
Puedes probarlo en \url{https://www.tablesgenerator.com/}.



%%%%%%%%%%%%%%%%%%%%%%%%%%%%%%%%%%%%%%%%%%%%%%%%%%%%%%%%%%%%%%%%%%%%%%%%%%%%%%%%
%%%%%%%%%%%%%%%%%%%%%%%%%%%%%%%%%%%%%%%%%%%%%%%%%%%%%%%%%%%%%%%%%%%%%%%%%%%%%%%%
% DISEÑO E IMPLEMENTACIÓN %
%%%%%%%%%%%%%%%%%%%%%%%%%%%%%%%%%%%%%%%%%%%%%%%%%%%%%%%%%%%%%%%%%%%%%%%%%%%%%%%%

\cleardoublepage
\chapter{Diseño e implementación}
\label{sec:diseno}

Aquí viene todo lo que has hecho tú (tecnológicamente). 
Puedes entrar hasta el detalle. 
Es la parte más importante de la memoria, porque describe lo que has hecho tú.
Eso sí, normalmente aconsejo no poner código, sino diagramas.



\section{Arquitectura general} 
\label{sec:arquitectura}

Si tu proyecto es un software, siempre es bueno poner la arquitectura (que es cómo se estructura tu programa a ``vista de pájaro'').

\begin{figure}
  \centering
  \includegraphics[width=9cm, keepaspectratio]{img/arquitectura.png}
  \caption{Estructura del parser básico}
  \label{fig:arquitectura}
\end{figure}


Por ejemplo, puedes verlo en la figura~\ref{fig:arquitectura}.
\LaTeX \ pone las figuras donde mejor cuadran. 
Y eso quiere decir que quizás no lo haga donde lo hemos puesto\ldots 
Eso no es malo.
A veces queda un poco raro, pero es la filosofía de \LaTeX: tú al contenido, que yo me encargo de la maquetación.


 
Recuerda que toda figura que añadas a tu memoria debe ser explicada.
Sí, aunque te parezca evidente lo que se ve en la figura~\ref{fig:arquitectura}, la figura en sí solamente es un apoyo a tu texto.
Así que explica lo que se ve en la figura, haciendo referencia a la misma tal y como ves aquí.
Por ejemplo: En la figura~\ref{fig:arquitectura} se puede ver que la estructura del \emph{parser} básico, que consta de seis componentes diferentes: los datos se obtienen de la red, y según el tipo de dato, se pasará a un \emph{parser} específico y bla, bla, bla\ldots

Si utilizas una base de datos, no te olvides de incluir también un diagrama de entidad-relación.


%%%%%%%%%%%%%%%%%%%%%%%%%%%%%%%%%%%%%%%%%%%%%%%%%%%%%%%%%%%%%%%%%%%%%%%%%%%%%%%%
%%%%%%%%%%%%%%%%%%%%%%%%%%%%%%%%%%%%%%%%%%%%%%%%%%%%%%%%%%%%%%%%%%%%%%%%%%%%%%%%
% EXPERIMENTOS Y VALIDACIÓN %
%%%%%%%%%%%%%%%%%%%%%%%%%%%%%%%%%%%%%%%%%%%%%%%%%%%%%%%%%%%%%%%%%%%%%%%%%%%%%%%%

\cleardoublepage
\chapter{Experimentos y validación}
\label{chap:experimentos}

Este capítulo se introdujo como requisito en 2019. 
Describe los experimentos y casos de test que tuviste que implementar para validar tus resultados. 
Incluye también los resultados de validación que permiten afirmar que tus resultados son correctos. 


%%%%%%%%%%%%%%%%%%%%%%%%%%%%%%%%%%%%%%%%%%%%%%%%%%%%%%%%%%%%%%%%%%%%%%%%%%%%%%%%
%%%%%%%%%%%%%%%%%%%%%%%%%%%%%%%%%%%%%%%%%%%%%%%%%%%%%%%%%%%%%%%%%%%%%%%%%%%%%%%%
% RESULTADOS %
%%%%%%%%%%%%%%%%%%%%%%%%%%%%%%%%%%%%%%%%%%%%%%%%%%%%%%%%%%%%%%%%%%%%%%%%%%%%%%%%

\cleardoublepage
\chapter{Resultados}
\label{chap:resultados}

En este capítulo se incluyen los resultados de tu trabajo fin de grado.

Si es una herramienta de análisis lo que has realizado, aquí puedes poner ejemplos de haberla utilizado para que se vea su utilidad.


%%%%%%%%%%%%%%%%%%%%%%%%%%%%%%%%%%%%%%%%%%%%%%%%%%%%%%%%%%%%%%%%%%%%%%%%%%%%%%%%
%%%%%%%%%%%%%%%%%%%%%%%%%%%%%%%%%%%%%%%%%%%%%%%%%%%%%%%%%%%%%%%%%%%%%%%%%%%%%%%%
% CONCLUSIONES %
%%%%%%%%%%%%%%%%%%%%%%%%%%%%%%%%%%%%%%%%%%%%%%%%%%%%%%%%%%%%%%%%%%%%%%%%%%%%%%%%

\cleardoublepage
\chapter{Conclusiones}
\label{chap:conclusiones}


\section{Consecución de objetivos}
\label{sec:consecucion-objetivos}

Esta sección es la sección espejo de las dos primeras del capítulo de objetivos, donde se planteaba el objetivo general y se elaboraban los específicos.

Es aquí donde hay que debatir qué se ha conseguido y qué no. 
Cuando algo no se ha conseguido, se ha de justificar, en términos de qué problemas se han encontrado y qué medidas se han tomado para mitigar esos problemas.

Y si has llegado hasta aquí, siempre es bueno pasarle el corrector ortográfico, que las erratas quedan fatal en la memoria final.
Para eso, en Linux tenemos aspell, que se ejecuta de la siguiente manera desde la línea de \emph{shell}:

\begin{verbatim}
  aspell --lang=es_ES -c memoria.tex
\end{verbatim}

\section{Aplicación de lo aprendido}
\label{sec:aplicacion}

Aquí viene lo que has aprendido durante el Grado/Máster y que has aplicado en el TFG/TFM.
Una buena idea es poner las asignaturas más relacionadas y comentar en un párrafo los conocimientos y habilidades puestos en práctica.

\begin{enumerate}
  \item a
  \item b
\end{enumerate}


\section{Lecciones aprendidas}
\label{sec:lecciones_aprendidas}

Aquí viene lo que has aprendido en el Trabajo Fin de Grado/Máster.

\begin{enumerate}
  \item Aquí viene uno.
  \item Aquí viene otro.
\end{enumerate}


\section{Trabajos futuros}
\label{sec:trabajos_futuros}

Ningún proyecto ni software se termina, así que aquí vienen ideas y funcionalidades que estaría bien tener implementadas en el futuro.

Es un apartado que sirve para dar ideas de cara a futuros TFGs/TFMs.


%%%%%%%%%%%%%%%%%%%%%%%%%%%%%%%%%%%%%%%%%%%%%%%%%%%%%%%%%%%%%%%%%%%%%%%%%%%%%%%%
%%%%%%%%%%%%%%%%%%%%%%%%%%%%%%%%%%%%%%%%%%%%%%%%%%%%%%%%%%%%%%%%%%%%%%%%%%%%%%%%
% APÉNDICE(S) %
%%%%%%%%%%%%%%%%%%%%%%%%%%%%%%%%%%%%%%%%%%%%%%%%%%%%%%%%%%%%%%%%%%%%%%%%%%%%%%%%

\cleardoublepage
\appendix
\chapter{Manual de usuario}
\label{app:manual}

Esto es un apéndice.
Si has creado una aplicación, siempre viene bien tener un manual de usuario.
Pues ponlo aquí.

%%%%%%%%%%%%%%%%%%%%%%%%%%%%%%%%%%%%%%%%%%%%%%%%%%%%%%%%%%%%%%%%%%%%%%%%%%%%%%%%
%%%%%%%%%%%%%%%%%%%%%%%%%%%%%%%%%%%%%%%%%%%%%%%%%%%%%%%%%%%%%%%%%%%%%%%%%%%%%%%%
% BIBLIOGRAFIA %
%%%%%%%%%%%%%%%%%%%%%%%%%%%%%%%%%%%%%%%%%%%%%%%%%%%%%%%%%%%%%%%%%%%%%%%%%%%%%%%%

\cleardoublepage

% Las siguientes dos instrucciones es todo lo que necesitas
% para incluir las citas en la memoria
\bibliographystyle{abbrv}
\bibliography{memoria}  % memoria.bib es el nombre del fichero que contiene
% las referencias bibliográficas. Abre ese fichero y mira el formato que tiene,
% que se conoce como BibTeX. Hay muchos sitios que exportan referencias en
% formato BibTeX. Prueba a buscar en http://scholar.google.com por referencias
% y verás que lo puedes hacer de manera sencilla.
% Más información: 
% http://texblog.org/2014/04/22/using-google-scholar-to-download-bibtex-citations/

\end{document}
